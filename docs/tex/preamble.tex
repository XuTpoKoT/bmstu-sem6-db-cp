\usepackage{enumitem}
\setenumerate[0]{label=\arabic*)} % Изменение вида нумерации списков
\renewcommand{\labelitemi}{---}

\usepackage{caption}
\captionsetup[table]{justification=raggedright,singlelinecheck=off} % Изменение подписей к таблицам
\captionsetup{labelsep=endash, justification=centering} % Настройка подписей float объектов
\captionsetup[figure]{name=Рисунок} % Изменение подписей к рисункам

\usepackage{geometry}
\geometry{a4paper,left=3.1cm,right=1.1cm,top=2cm,bottom=2cm,bindingoffset=0cm}
%\geometry{left=30mm}
%\geometry{right=15mm}
%\geometry{top=20mm}
%\geometry{bottom=20mm}
\setlength{\parindent}{1.25cm}
%\linespread{1.5}

%===========ОПЦИОНАЛЬНО==========
\addbibresource{main.bib}

%\usepackage[T1]{fontenc}
%\usepackage{lmodern}

\usepackage{multirow}
\usepackage{makecell}

\usepackage{ulem} % Нормальное нижнее подчеркивание
% Дополнительное окружения для подписей
\usepackage{array}
\newenvironment{signstabular}[1][1]{
	\renewcommand*{\arraystretch}{#1}
	\tabular
}{
	\endtabular
}
\def\arraybackslash{\let\\\tabularnewline}

\usepackage{pdfpages}
\usepackage{pgfplots}
\pgfplotsset{compat=1.9}
\usetikzlibrary{datavisualization}
\usetikzlibrary{datavisualization.formats.functions}
\usepackage{csvsimple} % генерим таблички из csv

\usepackage{graphicx}
\newcommand{\imgHeight}[3] {
	\begin{figure}[H]
		\center{\includegraphics[height=#1]{inc/img/#2}}
		\caption{#3}
		\label{img:#2}
	\end{figure}
}

\newcommand{\imgWid}[3] {
	\begin{figure}[H]
		\center{\includegraphics[width=#1]{inc/img/#2}}
		\caption{#3}
		\label{img:#2}
	\end{figure}
}

\newcommand{\imgScale}[3] {
	\begin{figure}[h!]
		\center{\includegraphics[scale=#1]{inc/img/#2}}
		\caption{#3}
		\label{img:#2}
	\end{figure}
}

\usepackage{listingsutf8}
\lstset{columns=fixed,basicstyle=\small,breaklines=true,inputencoding=utf8,keywordstyle=\bfseries\underbar,frame=single,tabsize=2,xleftmargin=20pt,xrightmargin=5pt}
%\lstset{language=sql}
\lstset{numbers=left,numberstyle=\small}

\lstset{extendedchars=\true}

% \usepackage{listings}
% \usepackage{xcolor}
% \lstset{ %
% 	basicstyle=\small\sffamily,			% размер и начертание шрифта для подсветки кода
% 	numbers=left,						% где поставить нумерацию строк (слева\справа)
% 	stepnumber=1,						% размер шага между двумя номерами строк
% 	numbersep=5pt,						% как далеко отстоят номера строк от подсвечиваемого кода
% 	frame=single,						% рисовать рамку вокруг кода
% 	tabsize=4,							% размер табуляции по умолчанию равен 4 пробелам
% 	captionpos=t,						% позиция заголовка вверху [t] или внизу [b]
% 	breaklines=true,					
% 	breakatwhitespace=true,				% переносить строки только если есть пробел
% 	escapeinside={\#*}{*)},				% если нужно добавить комментарии в коде
% 	backgroundcolor=\color{white},
% }
% Команда для римских цифр
%\newcommand{\rom}[1]{\MakeUppercase{\romannumeral #1}}
