\chapter{Конструкторский раздел}

\section{Описание сущностей базы данных}

Обозначения в таблицах: PK --- первичный ключ, FK --- внешний ключ, U --- поле должно быть уникальным, NN --- поле обязательно должно быть заполнено.

Описание сущностей базы данных приведено в таблицах \ref{tbl:user}--\ref{tbl:cart}.

\begin{table}[H]
	\begin{center}
		\caption{Таблица User}
		\label{tbl:user}
		\begin{tabular}{|c|c|c|c|}
			\hline
			поле & тип данных & ограничения & значение \\
			\hline
			login & text  & PK & логин \\
			\hline
			password & text & NN & пароль \\
			\hline
			email & text & U & электронная почта \\
			\hline
			role & text & NN & роль \\
			\hline
		\end{tabular}
	\end{center}	
\end{table}

\begin{table}[H]
	\begin{center}
		\caption{Таблица Card}
		\label{tbl:card}
		\begin{tabular}{|c|c|c|c|}
			\hline
			поле & тип данных & ограничения & значение \\
			\hline
			customer\_login & text & FK, NN & логин заказчика \\
			\hline
			bonuses & text & NN & количество бонусов \\
			\hline
		\end{tabular}
	\end{center}	
\end{table}

\begin{table}[H]
	\begin{center}
		\caption{Таблица Manufacturer}
		\label{tbl:man}
		\begin{tabular}{|c|c|c|c|}
			\hline
			поле & тип данных & ограничения & значение \\
			\hline
			id & uuid  & PK & идентификатор \\
			\hline
			name & text & NN & имя \\
			\hline
		\end{tabular}
	\end{center}	
\end{table}

\begin{table}[H]
	\begin{center}
		\caption{Таблица Product}
		\label{tbl:prod}
		\begin{tabular}{|c|c|c|c|}
			\hline
			поле & тип данных & ограничения & значение \\
			\hline
			id & text  & PK & идентификатор \\
			\hline
			name & text & NN & название \\
			\hline
			price & int & NN & цена \\
			\hline
			description & text & & описание \\
			\hline
			color & text & & цвет \\
			\hline
			storage\_cnt & int & NN & количество на складе \\
			\hline
			img\_ref & text & & ссылка на изображение \\
			\hline
			manufacturer\_id & uuid & FK, NN & идентификатор производителя \\
			\hline
			characteristics & json & & характеристики \\
			\hline
		\end{tabular}
	\end{center}	
\end{table}
\begin{table}[H]
	\begin{center}
		\caption{Таблица DeliveryPoint}
		\label{tbl:dp}
		\begin{tabular}{|c|c|c|c|}
			\hline
			поле & тип данных & ограничения & значение \\
			\hline
			id & uuid  & PK & идентификатор \\
			\hline
			address & text & U, NN & адрес \\
			\hline
		\end{tabular}
	\end{center}	
\end{table}

\begin{table}[H]
	\begin{center}
		\caption{Таблица Order}
		\label{tbl:order}
		\begin{tabular}{|c|c|m{30mm}|p{40mm}|}
			\hline
			поле & тип данных & ограничения & значение \\
			\hline
			id & uuid  & PK & идентификатор \\
			\hline
			customer\_login & text & FK & логин заказчика \\
			\hline
			employee\_login & text & FK & логин сотрудника \\
			\hline
			date & timestamp & NN & дата \\
			\hline
			status & text & NN & статус \\
			\hline
			delivery\_point\_id & uuid & NN & идентификатор точки доставки \\
			\hline
			initial\_cost & int & NN & начальная стоимость заказа \\
			\hline
			paid\_by\_bonuses & int & NN & оплачено бонусами \\
			\hline
		\end{tabular}
	\end{center}	
\end{table}

\begin{table}[H]
	\begin{center}
		\caption{Таблица Order\_Product}
		\label{tbl:op}
		\begin{tabular}{|c|c|c|c|}
			\hline
			поле & тип данных & ограничения & значение \\
			\hline
			order\_id & uuid  & FK, NN & идентификатор заказа \\
			\hline
			product\_id & uuid & FK, NN & идентификатор товара \\
			\hline
			price & int & NN & цена товара \\
			\hline
			cnt\_products & int & NN & количество товара \\
			\hline
		\end{tabular}
	\end{center}	
\end{table}

\begin{table}[H]
	\begin{center}
		\caption{Таблица Cart}
		\label{tbl:cart}
		\begin{tabular}{|c|c|c|c|}
			\hline
			поле & тип данных & ограничения & значение \\
			\hline
			product\_id & uuid & FK, NN & идентификатор товара \\
			\hline
			login & text & FK, NN & логин заказчика \\
			\hline
			cnt\_products & int & NN & количество товара \\
			\hline
		\end{tabular}
	\end{center}	
\end{table}


\section{Ролевая модель}
В данном разделе описаны права, которыми обладают пользователи на уровне базы данных.

Права неавторизованного пользователя:
\begin{itemize}
	\item выборка из таблиц User, Product, Manufacturer, DeliveryPoint, Card;
	\item вставка в таблицы Card, User.
\end{itemize}

Права заказчика:
\begin{itemize}
	\item выборка из таблиц User, Product, Manufacturer, DeliveryPoint, Card, Order\_Product;
	\item вставка в таблицы Order, Order\_Product, Cart.
	\item обновление таблиц Card, Product, Cart.
	\item удаление из таблицы Cart.
\end{itemize}

Права сотрудника:
\begin{itemize}
	\item выборка из таблиц User, Product, Manufacturer, DeliveryPoint, Card, Order\_Product;
	\item вставка в таблицы Order, Order\_Product, Cart.
	\item обновление таблиц Card, Product, Cart, Order.
	\item удаление из таблицы Cart.
\end{itemize}

Администратор обладает доступом к выполнению любых операций с любыми таблицами.

\section{Разработка триггера}
При выполнении операции вставки записи в таблицу заказов, в системе автоматически уменьшается количество товара на складе.
Для этого используется специальный триггер decrease\_storage\_cnt\_trigger. 
Триггер decrease\_storage\_cnt\_trigger относится к категории AFTER (после выполнения операции) и выполняет UPDATE запрос для изменения соответствующей записи в таблице Product.

\clearpage
На рисунке \ref{img:trig} представлена схема алгоритма работы триггера.

\imgHeight{150mm}{trig}{Схема алгоритма работы триггера}



