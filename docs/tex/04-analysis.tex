\chapter{Аналитический раздел}

\section{Анализ существующих решений}

Сформулируем критерии сравнения существующих решений:
\begin{itemize}
	\item К1: наличие программы лояльности;
	\item К2: возможность доставки;
	\item К3: возможность отслеживать текущий статус заказа;
	\item К4: доступность в РФ.
\end{itemize}

Результаты сравнения существующих решений с разрабатываемым приведены в таблице \ref{tabular:cmp}.
\begin{table}[h]
	\begin{center}
		\caption{\label{tabular:cmp} Сравнение существующих решений}
		\begin{tabular}{|c|c|c|c|c|c|}
			\hline
			Решение & К1 & К2 & К3 & К4\\
			\hline
			Музторг & + & + & + & + \\
			\hline
			Мир музыки & - & + & + & + \\
			\hline
			Music Store & + & + & + & - \\
			\hline
			Разрабатываемое решение & + & + & + & + \\
			\hline
		\end{tabular}
	\end{center}
\end{table}

Таким образом, разрабатываемое решение не уступает существующим по всем критериям.

\section{Описание пользователей приложения}
Требуется реализовать четыре вида ролей --- неавторизованный пользователь, заказчик, сотрудник магазина
и администратор.

Действия неавторизованного пользователя:
\begin{itemize}
	\item войти/зарегистрироваться;
	\item посмотреть ассортимент товаров;
	\item посмотреть информацию о конкретном товаре.
\end{itemize}

Действия заказчика:
\begin{itemize}
	\item выйти;
	\item посмотреть личные данные;
	\item посмотреть ассортимент товаров;
	\item посмотреть информацию о конкретном товаре;
	\item добавить товар в корзину;
	\item оформить заказ;
	\item посмотреть историю заказов.
\end{itemize}

Действия сотрудника:
\begin{itemize}
	\item выйти;
	\item посмотреть личные данные;
	\item посмотреть ассортимент товаров;
	\item посмотреть информацию о конкретном товаре;
	\item добавить товар в корзину;
	\item оформить заказ;
	\item изменить статус заказа;
	\item изменить количество товара на складе;	
	\item посмотреть историю продаж.
\end{itemize}

Действия администратора:
\begin{itemize}
	\item выйти;
	\item посмотреть ассортимент товаров;
	\item посмотреть информацию о конкретном товаре;
	\item изменить количество товара на складе;	
	\item добавить товар;
	\item добавить сотрудника.
\end{itemize}

\clearpage
На рисунке \ref{img:useCase} представлена диаграмма прецедентов для разных пользователей.
%\imgHeight{160mm}{useCase}{Диаграмма прецедентов}
\imgScale{0.45}{useCase}{Диаграмма прецедентов}



\section{Формализация данных}

База данных должна хранить информацию о:
\begin{itemize}
	\item пользователях;
	\item товарах;
	\item производителях;
	\item заказах;
	\item бонусных картах;
	\item точках доставки.
\end{itemize}

В таблице \ref{tbl:1} приведена информация о необходимых сущностях и их характеристиках.

\begin{table}[H]
	\begin{center}
	\caption{Сущности и характеристики}
	\label{tbl:1}
	\begin{tabular}{|c|p{100mm}|}
		\hline
		сущность & характеристики \\
		\hline
		пользователь & логин, пароль, email, ID бонусной карты, роль \\
		\hline
		товар & ID, название, цена, количество на складе, описание, цвет, производитель, ссылка на изображение \\
		\hline
		заказ & ID, логин заказчика, логин сотрудника, дата, статус, стоимость, количество потраченных бонусов, ID точки доставки\\
		\hline
		единица заказа & ID заказа, ID товара, количество товара, стоимость на момент заказа \\
		\hline
		точка доставки & ID, адрес \\
		\hline
		бонусная карта& ID, количество бонусов \\
		\hline
	\end{tabular}
	\end{center}
\end{table}

\clearpage
На рисунке \ref{img:er} отображена ER-диаграмма системы, основанная на приведенной выше таблице.

\imgHeight{130mm}{er}{ER-диаграмма}

\section{Выбор модели данных}

\subsection{Иерархическая модель}
В иерархической модели данные представляются в виде древовидной структуры.
Каждый объект в этой модели может являться предком по отношению к одному или нескольким дочерним узлам. При этом у дочернего узла может быть только один предок. 
Если удаляется предкок, то удаляются все дочерние узлы.
Особенностями модели являются быстрый доступ к данным за счет древовидной структуры и невозможность реализовать связь <<многие ко многим>>~\cite{s2}.

\subsection{Сетевая модель}
Сетевая модель данных основана на графовой структуре. 
Отличие от иерархической модели состоит в том, что в сетевой структуре данных у дочернего узла может быть любое число предков. Данная модель требует заранее определенной структуры данных, что может быть неудобно при изменениях в схеме базы данных. К особенностям сетевой модели можно отнести возможность реализовать сложные связи между данными и зависимость времени доступа к данным от их физической организации~\cite{s2}.

\subsection{Реляционная модель}
Реляционная модель данных основана на понятии отношения --- информационной модели реального объекта
предметной области, формально представленной множеством однотипных кортежей.
Кортеж отношения соответствует экземпляру объекта, свойства которого определяются значениями соответствующих атрибутов (полей) кортежа.
Кортежи отношений могут быть связаны между собой с помощью внешних ключей --- ссылок на соответствующие атрибуты~\cite{s2}.

Реляционная модель состоит из трех частей: структурной, целостностной и манипуляционной.
Структурная часть реляционной модели описывает, из каких объектов состоит реляционная модель.
Целостная часть определяет базовые требования целостности.
Манипуляционная часть описывает способы манипулирования данными.

\subsection{Модель ключ-значение}
В этой модели данные хранятся как совокупность пар ключ-значение, в которых ключ является уникальным и используется для поиска, добавления или удаления соответствующего значения.
При этом ключи могут являться сложными объектами.
В данной модели отсутстуют взаимосвязи между объектами, ограничения целостности.
Поиск по значениям неключевых атрибутов требуется выполнять на уровне приложения~\cite{s1}.

\subsection{Документо-ориентированная модель}
Базовая структурная единица документо-ориентированной модели --- это документ.
Схожие документы могут относиться к одной коллекции.
Если в реляционной модели все кортежи обладают идентичной структурой, то в документо-ориентированной базе данных все документы могут быть произвольными.
При использовании этой модели отсутствуют затраты на сборку данных из нескольких таблиц в единое целое для формирования результата запроса~\cite{s1}.

\section*{Вывод}
Для решения поставленной задачи была выбрана реляционная модель данных, так как она 
обеспечивает целостность на уровне базы данных и позволяет  организовать связь <<многие ко многим>>.
\clearpage

