\chapter*{ВВЕДЕНИЕ}
\addcontentsline{toc}{chapter}{ВВЕДЕНИЕ}

Российский рынок музыкальных инструментов потенциально является одним из самых крупных в Европе, о чем свидетельствует стабильный рост музыкальной индустрии в течение последних лет.
Ежегодный рост объема продаж музыкальных инструментов и оборудования по данным агентства «Бизнес Монитор», составляет 20--30~\%~\cite{intro}. В связи с этим, появляется спрос на программное обеспечение,
способное автоматизировать работу музыкальных магазинов.

Цель курсовой работы --- разработка базы данных для хранения и обработки данных музыкального магазина.

Для достижения цели необходимо выполнить следующие задачи:
\begin{itemize}
	\item провести анализ существующих решений;
	\item cформулировать описание пользователей приложения;
	\item формализовать данные;
	\item провести анализ моделей данных в базах данных;
	\item спроектировать архитектуру базы данных и ограничения целостности;
	\item спроектировать ролевую модель на уровне базы данных;
	\item выбрать средства реализации;
	\item реализовать спроектированную БД и необходимый интерфейс для взаимодействия с ней;
	\item сравнить время выполнения запросов на выборку при наличии индекса в базе данных и без него.
\end{itemize}
	
