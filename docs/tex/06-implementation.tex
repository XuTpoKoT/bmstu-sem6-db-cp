\chapter{Технологический раздел}

\section{Используемое программное обеспечение}
В качестве СУБД была выбрана реляционная система управления базами
данных PostgreSQL~\cite{postgres}.
PostgreSQL является бесплатной и доступной в РФ, поддерживается большинством операционных систем, имеет открытый исходный код.
Кроме того, данная СУБД позволяет обеспечить целостность данных с помощью различных ограничений.

Для реализации приложения был выбран платформонезависимымй язык программирования Java~\cite{java}, cистема автоматической сборки Gradle~\cite{gradle} и набор программного обеспечения \mbox{openjdk-20}, содержащий библиотеку JDBC~\cite{jdbc}, позволяющую работать с реляционными базами данных.

\section{Реализация сущностей системы}
В листингах \ref{lst:er}--\ref{lst:er3} приведены запросы для создания таблиц.

\begin{lstlisting}[label=lst:er,caption=Запросы для создания таблиц (часть 1)]
CREATE TABLE IF NOT EXISTS public.User (
	login 	     text  PRIMARY KEY DEFAULT gen_random_uuid()
	, password   bytea NOT NULL
	, role_      text  NOT NULL
	, first_name text
	, last_name  text
	, birth_date date
	, email      text
	, constraint check_role check (role_ in ('EMPLOYEE','CUSTOMER', 'ADMIN'))
);

CREATE TABLE IF NOT EXISTS public.Card (
	user_login     text NOT NULL REFERENCES public.User(login) ON DELETE CASCADE
	, bonuses 	   int  NOT NULL DEFAULT 0 CHECK (bonuses >= 0)
);
CREATE TABLE IF NOT EXISTS public.Manufacturer (
	id           uuid PRIMARY KEY DEFAULT gen_random_uuid()
	, name_      text UNIQUE NOT NULL
);
\end{lstlisting}

\begin{lstlisting}[label=lst:er2,caption=Запросы для создания таблиц (часть 2)]
CREATE TABLE IF NOT EXISTS public.Product (
	id 	              uuid PRIMARY KEY DEFAULT gen_random_uuid()
	, name_           text NOT NULL
	, price           int  NOT NULL
	, description     text
	, color           text
	, storage_cnt     int  NOT NULL DEFAULT 0 CHECK (storage_cnt >= 0)
	, img_ref         text
	, manufacturer_id uuid NOT NULL REFERENCES public.Manufacturer(id) ON DELETE CASCADE
	, characteristics json
);

CREATE TABLE IF NOT EXISTS public.DeliveryPoint (
	id            uuid PRIMARY KEY DEFAULT gen_random_uuid()
	, address 	  text UNIQUE NOT NULL
);

CREATE TABLE IF NOT EXISTS public.Order_(
	id                 uuid        PRIMARY KEY DEFAULT gen_random_uuid()
	, customer_login   text        REFERENCES public.User(login) ON DELETE CASCADE
	, employee_login   text        REFERENCES public.User(login) ON DELETE CASCADE
	, date_            timestamptz NOT NULL
	, status           text        NOT NULL
	, delivery_point_id uuid       NOT NULL REFERENCES public.DeliveryPoint(id) ON DELETE CASCADE
	, initial_cost     int         NOT NULL CHECK (initial_cost > 0)
	, paid_by_bonuses  int         NOT NULL CHECK (paid_by_bonuses <= initial_cost)
	, constraint check_status check (status in ('formed','built', 'delivered', 'received'))
);
\end{lstlisting}

\clearpage
\begin{lstlisting}[label=lst:er3,caption=Запросы для создания таблиц (часть 3)]
CREATE TABLE IF NOT EXISTS public.Order_Product (
	order_id           uuid NOT NULL REFERENCES public.Order_(id)  ON DELETE CASCADE
	, product_id       uuid NOT NULL REFERENCES public.Product(id) ON DELETE CASCADE
	, price      	   int  NOT NULL CHECK (price > 0)
	, cnt_products 	   int  NOT NULL CHECK (cnt_products > 0)
);

CREATE TABLE IF NOT EXISTS public.Cart (
	login              text NOT NULL REFERENCES public.User(login) ON DELETE CASCADE
	, product_id       uuid NOT NULL REFERENCES public.Product(id) ON DELETE CASCADE
	, cnt_products 	   int  NOT NULL CHECK (cnt_products > 0)
	, primary key(login, product_id)
);
\end{lstlisting}

\section{Реализация триггера}
В листинге \ref{lst:trig1} приведена реализация функции decrease\_storage\_cnt.
%Был разработан спроектированный триггер AFTER на запрос DELETE в
%аблицу Brand. 
\begin{lstlisting}[label=lst:trig1,caption=Реализация функции decrease\_storage\_cnt]
CREATE OR REPLACE FUNCTION decrease_storage_cnt()
	RETURNS TRIGGER AS
$$
BEGIN
	update public.product
	SET storage_cnt = storage_cnt - new.cnt_products
	where id = new.product_id;
	RETURN new;
END;
$$ language plpgsql;
\end{lstlisting}

\clearpage
В листинге \ref{lst:trig2} приведена реализация триггера decrease\_storage\_cnt\_trigger.
\begin{lstlisting}[label=lst:trig2,caption=Создание триггера decrease\_storage\_cnt\_trigger]
CREATE TRIGGER decrease_storage_cnt_trigger
	AFTER INSERT
	ON public.order_product
	FOR EACH ROW
EXECUTE PROCEDURE decrease_storage_cnt();
\end{lstlisting}

\section{Реализация ролевой модели}
В листингах \ref{lst:r1}--\ref{lst:r4} приведена реализация ролевой модели.
\begin{lstlisting}[label=lst:r1,caption=Реализация роли заказчика]
CREATE ROLE customer;
GRANT SELECT ON public.User TO customer;
GRANT SELECT ON public.Manufacturer TO customer;
GRANT SELECT, UPDATE ON public.Product TO customer;
GRANT SELECT ON public.DeliveryPoint TO customer;
GRANT SELECT, UPDATE ON public.Card TO customer;
GRANT SELECT, INSERT ON public.Order_ TO customer;
GRANT SELECT, INSERT ON public.Order_Product TO customer;
GRANT SELECT, INSERT, update, Delete ON public.Cart TO customer;
\end{lstlisting}

\begin{lstlisting}[label=lst:r2,caption=Реализация роли сотрудника]
CREATE ROLE employee;
GRANT SELECT ON public.User TO employee;
GRANT SELECT ON public.Manufacturer TO employee;
GRANT SELECT, UPDATE ON public.Product TO employee;
GRANT SELECT ON public.DeliveryPoint TO employee;
GRANT SELECT, UPDATE ON public.Card TO employee;
GRANT SELECT, INSERT, UPDATE ON public.Order_ TO employee;
GRANT SELECT, INSERT ON public.Order_Product TO employee;
GRANT SELECT, INSERT, update, Delete ON public.Cart TO employee;
\end{lstlisting}

\begin{lstlisting}[label=lst:r3,caption=Реализация роли администратора]
CREATE ROLE admin_;
grant all privileges ON all tables in schema public to admin_;
\end{lstlisting}

\clearpage
\begin{lstlisting}[label=lst:r4,caption=Реализация роли незарегистрированного пользователя]
CREATE ROLE unregistered;
GRANT SELECT, INSERT ON public.User TO unregistered;
GRANT SELECT ON public.Manufacturer TO unregistered;
GRANT SELECT ON public.Product TO unregistered;
GRANT SELECT ON public.DeliveryPoint TO unregistered;
GRANT INSERT ON public.Card TO unregistered;
\end{lstlisting}

\section{Демонстрация работы приложения}
На рисунках \ref{img:main}--\ref{img:user} изображена главная страница, страница  с детельной информацией о товаре, а также страница с информацией о пользователе.
продемонстрирован интерфейс приложения.

\imgWid{140mm}{main}{Главная страница}

\imgWid{140mm}{product}{Страница товара}

\imgWid{140mm}{user}{Информация о пользователе}

На рисунках \ref{img:cart}--\ref{img:orders} изображена корзина пользователя и его история заказов.
\imgWid{140mm}{cart}{Корзина пользователя}
\imgWid{140mm}{orders}{История заказов}