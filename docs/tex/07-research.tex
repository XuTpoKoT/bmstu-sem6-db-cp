\chapter{Исследовательская часть}

\section{Постановка исследования}
Цель исследования --- сравнение времени выполнения запросов на выборку при наличии индекса в базе данных и без него.

В листинге \ref{lst:query} приведен запрос на выборку из таблицы товаров по названию товара.
\begin{lstlisting}[label=lst:query,caption=запрос на выборку из таблицы товаров]
	SELECT p.id
	FROM public.product p
	WHERE p.name_ = ?;
\end{lstlisting}

В листинге \ref{lst:ind} приведен запрос для создания индекса на поле name\_ таблицы товаров.
\begin{lstlisting}[label=lst:ind,caption=запрос для создания индекса]
CREATE UNIQUE INDEX product_name_idx ON public.product (name_);
\end{lstlisting}
Технические характеристики устройства, на котором выполнялись замеры времени:

\begin{itemize}
	\item операционная система --- Ubuntu 22.04.1 Linux x86\_64;
	\item оперативная память --- 8 ГБ;
	\item процессор --- AMD Ryzen 5 3550H \cite{amd}.
\end{itemize}

Замеры проводились на ноутбуке, включенном в сеть электропитания. Во время замеров ноутбук не был нагружен сторонними приложениями.

\section{Результаты исследования}

На рисунке \ref{img:g1} представлен график, иллюстрирующий зависимость времени выполнения запроса на выборку из таблицы товаров от количества записей и от наличия индекса.

\clearpage
\begin{figure}[h!]
	\centering
	\begin{tikzpicture}
		\begin{axis}[
			height = 0.4\paperheight, 
			width = 0.65\paperwidth,
			legend pos = north west,
			table/col sep=comma,
			xlabel={кол-во записей в таблице},
			ylabel={время, мкс},
			]
			\legend{ 
				Без индексов, 
				С индексами, 
			};
			\addplot [
			solid, 
			draw = blue,
			mark = *, 
			mark options = {
				scale = 1.5, 
				fill = blue, 
				draw = black
			}
			] table [x={size}, y={time}] {inc/csv/usual.csv};
			\addplot [
			dotted, 
			draw = red,
			mark = star,
			mark options = {
				scale = 1.5, 
				draw = red
			}
			] table [x={size}, y={time}] {inc/csv/indexes.csv};	
		\end{axis}
	\end{tikzpicture}
	\caption{Сравнение времени выполнения запросов на выборку при наличии индекса и без него}
	\label{img:g1}
\end{figure}

\section*{Вывод}
При 100 записях в таблице время выполнения запроса при наличии индекса быстрее на 15~\%, при 1000 записях --- на 31~\%.
Таким образом, индексы значительно ускоряют операции выборки.
Однако важно отметить, что индексы замедляют операции вставки, обновления и удаления данных, так как база данных должна поддерживать актуальность индекса. 
